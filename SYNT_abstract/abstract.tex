\documentclass[submission]{eptcs}
\providecommand{\event}{SYNT 2018} % Name of the event you are submitting to
\usepackage{breakurl}             % Not needed if you use pdflatex only.
\usepackage{underscore}           % Only needed if you use pdflatex.

\title{Safe, Automated and Formal Synthesis of Digital Controllers 
for Continuous Plants}
%\author{Elizabeth Polgreen
%\institute{NICTA\\ Sydney, Australia}
%\institute{School of Computer Science and Engineering\\
%University of New South Wales\thanks{A fine university.}\\
%Sydney, Australia}
%\email{rvg@cs.stanford.edu}
%\and
%Co Author \qquad\qquad Yet S. Else
%\institute{Stanford Univeristy\\
%California, USA}
%\email{\quad is@gmail.com \quad\qquad somebody@else.org}
%}
%\def\titlerunning{Synthesis of Digital Controllers}
%\def\authorrunning{R.J. van Glabbeek, C. Author \& Y.S. Else}
\begin{document}
\maketitle

\begin{abstract}
We present a sound and automated approach to synthesizing safe,
digital controllers for physical plants represented as linear,
time-invariant models. The synthesis
accounts for errors caused by the digitization effects introduced by
digital controllers operating in either fixed- or floating-point arithmetic. 
Our approach uses counterexample-guided inductive
synthesis (CEGIS): in the first phase an inductive generalisation engine produces a
possible solution that is safe for some possible initial conditions but may
not be safe for all initial conditions. Safety for all initial conditions
is then verified in a second phase either
via bounded model checking or abstract acceleration; if the verification step fails, a
counterexample is provided to the inductive generalisation and the
process iterates until a safe controller is obtained.  We implemented our approach
in a tool named DSSynth (Digital-System Synthesizer) and demonstrate
its practical value by automatically synthesizing safe controllers for physical 
plant models from the digital control literature.
\end{abstract}

\section{Introduction}

Embedded control systems using fixed and floating-point arithmetic 
have become widespread
as the availability of low-cost devices that can perform highly
non-trivial control tasks has increased. Correct synthesis of
control software for such platforms is non-trivial due to the digital representation of continuous quantities 
that is used by the controller. This digital representation introduces 
errors due to finite-precision arithmetic, time discretization and 
A/D - D/A conversions.


%
Given an LTI model of a physical (continuous) plant, we present two automated approaches for generating
correct-by-construction digital controllers that address all these
challenges and satisfy a safety property for the physical plant. 
Both approaches make use of CounterExample-Guided
Inductive Synthesis (CEGIS)~\cite{jha-icse10,
  DBLP:conf/asplos/Solar-LezamaTBSS06}.  CEGIS is an
iterative process, where each iteration performs inductive
generalisation based on counterexamples provided by a verification module. 
The inductive generalisation step uses information 
about a limited number of inputs to compute a candidate solution
for all the possible inputs. Our two instantiations of CEGIS are
described next.


\emph{The first approach} starts by devising a
digital controller that stabilizes the system's model, while remaining safe for a
pre-selected time horizon and a single initial state; then, it verifies
unbounded-time safety by unfolding the dynamics of the LTI model, considering the
full set of initial states, and checking a \emph{completeness
threshold}~\cite{DBLP:conf/vmcai/KroeningS03}, i.e., the number of
iterations required to sufficiently unwind the closed-loop state-space
model such that the boundaries are not violated for any larger number of
iterations.  As~it requires unfolding up to the completeness threshold, this
approach can be computationally expensive if the completeness threshold is large.
%
\emph{The second approach}
employs \emph{abstract acceleration}~\cite{cattaruzza2015unbounded} to
evaluate all possible progressions of the LTI model simultaneously. 
This approach uses \emph{abstraction refinement},
enabling us to always start with a very simple description regardless of the
dynamics complexity, and only expand to more complex models
when a solution cannot be found.

We provide experimental results showing that both our approaches are able to
efficiently synthesize safe controllers for a set of intricate physical
plant models taken from the digital control literature.

This abstract contains material published in \cite{DBLP:conf/kbse/AbateBCCCDKKP17,DBLP:conf/cav/AbateBCCDKKP17}.

\bibliographystyle{eptcs}       
\bibliography{../automatica/paper}  
\end{document}
