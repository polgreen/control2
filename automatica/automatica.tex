% autosam.tex
% Annotated sample file for the preparation of LaTeX files
% for the final versions of papers submitted to or accepted for 
% publication in AUTOMATICA.

% See also the Information for Authors.

% Make sure that the zip file that you send contains all the 
% files, including the files for the figures and the bib file.

% Output produced with the elsart style file does not imitate the
% AUTOMATICA style. The style file is generic for all Elsevier
% journals and the output is laid out for easy copy editing. The
% final document is produced from the source file in the
% AUTOMATICA style at Elsevier.

% You may use the style file autart.cls to obtain a two-column 
% document (see below) that more or less imitates the printed 
% Automatica style. This may helpful to improve the formatting 
% of the equations, tables and figures, and also serves to check 
% whether the paper satisfies the length requirements.

% Please note: Authors must not create their own macros.

% For further information regarding the preparation of LaTeX files 
% for Elsevier, please refer to the "Full Instructions to Authors" 
% from Elsevier's anonymous ftp server on ftp.elsevier.nl in the
% directory pub/styles, or from the internet (CTAN sites) on
% ftp.shsu.edu, ftp.dante.de and ftp.tex.ac.uk in the directory
% tex-archive/macros/latex/contrib/supported/elsevier.


%\documentclass{elsart}               % The use of LaTeX2e is preferred.

\documentclass[twocolumn]{autart}    % Enable this line and disable the 
                                     % preceding line to obtain a two-column 
                                     % document whose style resembles the
                                     % printed Automatica style.


\usepackage{graphicx}          % Include this line if your 
                               % document contains figures,
%\usepackage[dvips]{epsfig}    % or this line, depending on which
                               % you prefer.
\usepackage{color}
\usepackage{amssymb}
\usepackage{amsmath}
\setcounter{tocdepth}{3}
\usepackage{graphicx}
\usepackage{tikz}
\usepackage{pgfplots}
\usepackage{framed}
\usepackage{listings}
\usepackage{algorithm,algpseudocode}
\usepackage{epstopdf}
\usepackage{cite}
\usepackage{pifont}
\usepackage{todonotes}
\usetikzlibrary{positioning, automata, shapes.arrows, calc, shapes, arrows}
\usetikzlibrary{patterns}
\usepackage{url}

\begin{document}

\begin{frontmatter}
%\runtitle{Insert a suggested running title}  % Running title for regular 
                                              % papers but only if the title  
                                              % is over 5 words. Running title 
                                              % is not shown in output.

%\thanksref{footnoteinfo}

\title{Safe and Robust Formal  Synthesis of Digital Controllers for Continuous Plants with Transient Performance Specifications} % Title, preferably not more 
                                                % than 10 words.

%\thanks[footnoteinfo]{This paper was not presented at any IFAC 
%meeting. Corresponding author M.~T.~Cicero. Tel. +XXXIX-VI-mmmxxi. 
%Fax +XXXIX-VI-mmmxxv.}

\author[oxford]{Alessandro Abate}\ead{alessandro.abate@cs.ox.ac.uk},
\author[manaus]{Iury Bessa}\ead{iurybessa@ufam.edu.br},
\author[oxford]{Dario Cattaruzza}\ead{dario.cattaruzza@cs.ox.ac.uk},
\author[oxford,manaus]{Lucas Cordeiro}\ead{lucas.cordeiro@cs.ox.ac.uk},
\author[oxford]{Cristina David}\ead{cristina.david@cs.ox.ac.uk},
\author[oxford]{Pascal Kessel}\ead{pascal.kesseli@stx.ox.ac.uk},
\author[oxford]{Daniel Kroening}\ead{kroening@cs.ox.ac.uk},
\author[oxford]{Elizabeth Polgreen}\ead{elizabeth.polgreen@linacre.ox.ac.uk}

\address[oxford]{University of Oxford, UK}
\address[manaus]{Federal University of Amazonas, Brazil}

%\author[Paestum]{Marcus Tullius Cicero}\ead{cicero@senate.ir},    % Add the 
%\author[Rome]{Julius Caesar}\ead{julius@caesar.ir},               % e-mail address 
%\author[Baiae]{Publius Maro Vergilius}\ead{vergilius@culture.ir}  % (ead) as shown

%\address[Paestum]{Buckingham Palace, Paestum}  % Please supply                                              
%\address[Rome]{Senate House, Rome}             % full addresses
%\address[Baiae]{The White House, Baiae}        % here.

          
\begin{keyword}                           % Five to ten keywords,  
Digital Control; A/D converters; Control System Synthesis; Transient Analysis; Safety Analysis; Quantization Errors; Sampling.               % chosen from the IFAC 
\end{keyword}                             % keyword list or with the 
                                          % help of the Automatica 
                                          % keyword wizard

%\keywords{
%State-space dynamical models of physical systems; 
%digital controllers; 
%analogue-to-digital converters; 
%time sampling; 
%quantization; 
%fixed-point arithmetic; 
%CEGIS; 
%safety requirements. 
%}


\begin{abstract}                          % Abstract of not more than 200 words.
\textcolor{red}{We have to write a new abstract.}
\end{abstract}

\end{frontmatter}

\section{Introduction}
\textcolor{red}{We have to write a new introduction as well.} 

%\begin{figure}
%\begin{center}
%\includegraphics[height=4cm]{jcaesar.eps}    % The printed column  
%\caption{Gaius Julius Caesar, 100--44 B.C.}  % width is 8.4 cm.
%\label{fig1}                                 % Size the figures 
%\end{center}                                 % accordingly.
%\end{figure}

% OR

%\begin{figure}
%\begin{center}
%\epsfig{file=jcaesar,width=7cm}
%\caption{Gaius Julius Caesar, 100--44 B.C.}
%\label{fig1}
%\end{center}
%\end{figure}


%\subsection{A subsection}
%Marcus Tullius Cicero, 106--43 B.C. was a Roman statesman, orator, 
%and philosopher.  A major figure in the last years of the Republic, 
%he is best known for his orations against Catiline\footnote{
%This footnote should be very brief.}
%and for his mastery of Latin prose \cite{Heritage:92}. He was a 
%contemporary of Julius Caesar (Fig.~\ref{fig1}).

%-------------------------------
\section{Related Work}
\label{sec:relw}
%-------------------------------

\textbf{CEGIS -}
Program synthesis is the problem of computing correct-by-design programs
from high-level specifications. Algorithms for this problem have made
substantial progress in recent years, for instance~\cite{itzhaky2010simple} 
to inductively synthesize invariants for the generation of desired programs.

Program synthesizers are an ideal fit for the synthesis of digital controllers, since
the semantics of programs capture the effects of finite-precision arithmetic
precisely.  In~\cite{DBLP:conf/cdc/RavanbakhshS15}, the authors use CEGIS
for the synthesis of switching controllers for stabilizing continuous-time
plants with polynomial dynamics.  The work extends to affine systems, but is
limited by the capacity of the state-of-the-art SMT solvers for solving
linear arithmetic.  Since this approach uses switching models instead of
linear dynamics for the digital controller, it avoids problems related to
finite precision arithmetic, but potentially suffers from state-space
explosion.  Moreover, in \cite{DBLP:conf/emsoft/RavanbakhshS16} the same
authors use a CEGIS-based approach for synthesizing continuous-time
switching controllers that guarantee \emph{reach-while-stay} properties of
closed-loop systems, i.e., properties that specify a set of goal states and
safe states (constrained reachability).  This solution is based on
synthesizing control Lyapunov functions for switched systems that yield
switching controllers with a guaranteed minimum dwell time in each mode. 
However, both approaches are unsuitable for the kind of control we seek to
synthesize.

The work in~\cite{hscc-paper} synthesizes stabilizing
controllers for continuous plants given as transfer functions by exploiting
bit-accurate verification of software-implemented digital
controllers~\cite{Bessa16}.  While this work also uses CEGIS,
the approach is restricted to digital controllers for stable closed-loop
systems given as transfer function models: 
this results in  a static check on their coefficients.  
By contrast, in the current paper we consider a state-space representation of the physical system, 
which requires ensuring the specification over actual traces of the model, 
alongside the numerical soundness required by the effects of discretisation and finite-precision errors.   
A state-space model has known advantages over the transfer function
representation~\cite{Franklin15}: it naturally generalizes to multivariate systems
(i.e., with multiple inputs and outputs); 
and it allows synthesis of control systems with guarantees on the internal dynamics, e.g.,
to synthesize controllers that make the closed-loop system \emph{safe}.  Our
work focuses on the \emph{safety} of internal states, which is usually
overlooked in the literature.  Moreover, our work integrates an
abstraction/refinement (CEGAR) step inside the main CEGIS loop.

The tool Pessoa~\cite{mazo2010pessoa} synthesizes correct-by-design embedded
control software in a Matlab toolbox.  It is based on the abstraction of a
physical system to an equivalent finite-state machine and on the computation
of reachability properties thereon. 
Based on this safety specification, \mbox{Pessoa} can synthesize embedded controller
software for a range of properties.  The embedded controller software can be
more complicated than the state-feedback control we synthesize, and the
properties available cover more detail. 
However, relying on state-space discretization \mbox{Pessoa} is likely to incur in scalability limitations. 
Along this research line, \cite{Anta2010,liu16} studies the synthesis of digital controllers for continuous dynamics, 
and \cite{zamani2014} extends the approach to the recent setup of Network Control Systems. 

\textbf{Discretization Effects -}
The classical approach to control synthesis has often disregarded digitalization effects, 
whereas more recently modern techniques have focused on
different aspects of discretization, including delayed
response~\cite{Duggirala2015} and finite word length (FWL) semantics, 
with the goal either to verify (e.g.,~\cite{daes20161}) or to optimize
(e.g.,~\cite{oudjida2014design}) given implementations. 

There are two different problems that arise from FWL semantics.  The first
is the error in the dynamics caused by the inability to represent the exact
state of the physical system, while the second relates to rounding and saturation errors
during computation.  In~\cite{fialho1994stability}, a stability measure
based on the error of the digital dynamics ensures that the deviation
introduced by FWL does not make the digital system unstable.  A~more recent
approach~\cite{DBLP:journals/automatica/WuLCC09} uses $\mu$-calculus to
directly model the digital controller so that the selected parameters are
stable by design.  The analyses in~\cite{DBLP:conf/hybrid/RouxJG15,
DBLP:conf/hybrid/WangGRJF16} rely on an invariant computation on the
discrete system dynamics using Semi-Definite Programming (SDP).  While the
former uses bounded-input and bounded-output (BIBO) properties to determine
stability, the latter uses Lyapunov-based quadratic invariants.  In both
cases, the SDP solver uses floating-point arithmetic and soundness is
checked by bounding the error.  An alternative is~\cite{park2016scalable},
where the verification of given control code is performed against a known
model by extracting an LTI model of the code by symbolic execution:  
to account for rounding errors, an upper bound is introduced in the
verification phase.  The work in \cite{picasso2003stabilization}
introduces invariant sets as a mechanism
to bound the quantization error effect on stabilization as an invariant set
that always converges toward the controllable set.  Similarly,
\cite{liberzon2003hybrid} evaluates the quantization error dynamics
and bounds its trajectory to a known region over a finite time period. 
This technique works for both linear and non-linear systems.

%-------------------------------
\section{Preliminaries}
\label{sec:preliminaries}
%-------------------------------

%-------------------------------
\subsection{State-space representation of physical systems} 
\label{ssec:ssrepresentation}
%-------------------------------

We consider models of physical plants expressed as ordinary differential
equations (ODEs), which we assume to be controllable and under full state
information (i.e., we have access to all the model variables):
%
\begin{align}
\label{eq:ode}
\dot{x}(t) = Ax(t)+ B u(t), \quad x \in \mathbb{R}^{n}, u \in \mathbb{R}^m, \\ \nonumber A \in \mathbb{R}^{n \times n}, B \in \mathbb{R}^{n \times m}, 
\end{align}
%
where $t \in \mathbb R_0^+$, where $A$ and $B$ are matrices that fully
specify the continuous plant, and with initial states set as $x(0)$.  While
ideally we intend to work on the continuous-time plant, in this work
Eq.~\eqref{eq:ode} is soundly discretized in time~\cite{fadali} into
%
\begin{align}
\label{eq:plant}
x_{k+1} = A_d x_k+ B_d u_k
\end{align} 
%
where $k \in \mathbb N$ and $x_{0}=x(0)$ is the initial state. 
$A_d$ and $B_d$ denote the matrices that describe the discretized plant dynamics, whereas
$A$ and $B$ denote the continuous plant dynamics.  
We synthesize for requirements over this discrete-time domain. 
Later, we will address the issue of variable quantization, 
as introduced by the ADC/DAC conversion blocks 
(\textcolor{red}{we should include a figure that describes the control system model that we tackle}).



%-------------------------------
\subsection{Controller synthesis via state feedback}
\label{ssec:statefeedbackcontrol}
%-------------------------------

Models \eqref{eq:ode} and \eqref{eq:plant} depend on external non-determinism in the form of input signals $u (t)$ and  $u_k$, respectively. 
Feedback architectures can be employed to manipulate the properties and behaviors of the continuous process (the plant).   
We are interested in the synthesis of digital feedback control algorithms, 
as implemented on Field-Programmable Gate Arrays or Digital Signal Processors. 
The most basic feedback architecture is the state feedback one, 
where the control action $u_k$ (notice we work with the discretized signal) is computed by: 
%
\begin{equation}
\label{eq:controlaction}
u_k = r_{k} - K x_k. 
\end{equation}
%
Here, $K \in \mathbb{R}^{m \times n}$ is a state-feedback gain matrix, 
and $r_{k}$ is a reference signal (again digital).   
%
The closed-loop model then takes the form 
\begin{align}
\label{eq:closedloopss}
x_{k+1} = ( A_d - B_d K ) x_k + B_d r_k.
\end{align}
%
The gain matrix $K$ can be set so that the closed-loop discrete dynamics are
shaped as desired, for instance according to a specific stability goal or
around a specific dynamical behavior \cite{astrom1997computer}.  As argued
later in this work, we will target more complex objectives, such as
quantitative safety requirements, which are not typical in the digital
control literature.  Further, we will embrace the digital nature of the
controller, which manipulates quantized signals as discrete quantities represented with finite precision. 

%-------------------------------
\subsection{Stability of closed-loop systems}
\label{ssec:stability}
%-------------------------------

In this work we employ asymptotic stability in the CEGIS loop,  
as an objective for guessing controllers that are later proven sound over safety requirements.  
Asymptotic stability is a property that amounts to convergence of the model executions to an equilibrium point, 
starting from any states in a neighborhood of the point (see Figure~\ref{fig:ct} for the portrait of a stable execution, converging to the origin).  
In the case of linear systems as in~\eqref{eq:closedloopss}, 
considered with a zero reference signal, 
the equilibrium point of interest is the origin. 

A discrete-time LTI system as \eqref{eq:closedloopss} is asymptotically
stable if all the roots of its characteristic polynomial (i.e., the
eigenvalues of the closed-loop matrix $A_d - B_d K$) are inside the unity
circle of the complex plane, i.e., their absolute values are strictly less than
one~\cite{astrom1997computer} (this simple sufficient condition can be generalised, 
however this is not necessary in our work).  
In this paper, we express this stability specification $\phi_\mathit{stability}$ in terms of a check known as
\emph{Jury's criterion}~\cite{fadali}: this is an easy algebraic formula to
select the entries of matrix $K$ so that the closed-loop dynamics are shaped
as desired.

%-------------------------------
\subsection{Safety specifications for dynamical systems}
\label{ssec:safety}
%-------------------------------

We are not limited to the synthesis of digital stabilizing controllers -- a
well known task in the literature on digital control systems -- but target
safety requirements with an overall approach that is sound and automated. 
More specifically, we require that the closed-loop system
\eqref{eq:closedloopss} meets given safety specifications.  A safety
specification gives raise to a requirement on the states of the model, such
that the feedback controller (namely the choice of the gains matrix~$K$)
must ensure that the state never violates the requirement.  Note that a
stable, closed-loop system is not necessarily a safe system: indeed, the
state values may leave the safe part of the state space while they converge
to the equilibrium, which is typical in the case of oscillatory dynamics. 
In~this work, the safety property is expressed as:
%
\begin{equation}
\label{eq:safetyliteral}
\phi_\mathit{safety}\iff \forall k\ge 0.\, \bigwedge_{i=1}^{n}{\underline{x_{i}} \leq x_{i,k} \leq \overline{x_{i}}},
\end{equation}
where $\underline{x_{i}}$ and $\overline{x_{i}}$ are lower and upper bounds
for the $i$-th coordinate $x_{i}$ of state $x\in \mathbb R^n$ at the $k$-th
instant, respectively.  This means that the states will always be within an $n$-dimensional hyper-box.

Furthermore, it is practically relevant to consider the 
constraints $\phi_\mathit{input}$ on the input
signal $u_{k}$ and $\phi_\mathit{init}$ on the initial states $x_0$,
which we assume have given bounds:
$\phi_\mathit{input} = {\forall k.\underline{u} \leq u_{k} \leq \overline{u}} $, 
$\phi_\mathit{init} = \bigwedge_{i=1}^{n} \underline{x_{i,0}} \leq x_{i,0} \leq \overline{x_{i,0}}.$
For the former, this means that the control input might saturates in view of
physical constraints.

%-------------------------------
\subsection{Performance specifications for dynamical systems}
\label{ssec:performance}
%-------------------------------

\textcolor{red}{we should describe here about the performance specifications for dynamical systems.}

%+++++++++++++++++++++++++++++++++++++++++++++++++++++++++++++++++++++++++++++++
\subsection{Numerical representation and soundness} 
\label{sec:numeric_rep}
%+++++++++++++++++++++++++++++++++++++++++++++++++++++++++++++++++++++++++++++++

\textcolor{red}{this section needs to be updated to describe about floating-point arithmetic.}

The models we consider have two sources of error that are due to numerical  
representation.  The first is the numerical error introduced by the
fixed-point numbers employed to model the plant, i.e., to represent the
plant dynamics $A_d$, $B_d$ and $x_k$.  The second is the quantization error
introduced by the digital controller, which performs operations on
fixed-point numbers.  In this section we outline the notation for the
fixed-point representation of numbers, and briefly describe the errors
introduced.  A~formal discussion is in
%Appendix~\ref{appendix:numerical_errors}.

Let $\mathcal{F}_{\langle I,F \rangle}(x)$ denote a real number $x$
represented in a fixed point domain, with $I$ bits representing the integer
part and $F$ bits representing the decimal part.  The smallest number that
can be represented in this domain is $c_m=2^{-F}$.  Any mathematical
operations performed at the precision $\mathcal{F}_{\langle I,F \rangle}(x)$
will introduce errors, for which an upper bound can be
given~\cite{DBLP:conf/arith/BrainTRW15}.

We will use $\mathcal{F}_{\langle I_c,F_c \rangle}(x)$ to denote a real
number $x$ represented at the fixed-point precision of the controller, and
$\mathcal{F}_{\langle I_p,F_p \rangle}(x)$ to denote a real number $x$
represented at the fixed-point precision of the plant model ($I_c$ and $F_c$ are determined by the controller. We pick $I_p$ and $F_p$ for our synthesis such that $I_p \geq I_c$ and $\allowbreak F_p \geq F_c$).  Thus any mathematical operations in our modelled
digital controller will be in the range of $\mathcal{F}_{\langle I_c,F_c
\rangle}$, and all other calculations in our model will be carried out in the range of
$\mathcal{F}_{\langle I_p,F_p \rangle}$.  
The physical plant operates in the
reals, which means our verification phase must also account for the numerical error and quantization errors caused by representing the physical plant at the finite precision $\mathcal{F}_{\langle I_p,F_p \rangle}$.


\subsubsection{Effect on safety specification and stability}

Let us first consider the effect of the quantization errors on safety. 
Within the controller, state values are manipulated at low precision,
alongside the vector multiplication $Kx$.
The inputs are computed using the following equation: 
%
\begin{align*}
u_{k}&=-(\mathcal{F}_{\langle I_c,F_c \rangle}(K)\cdot\mathcal{F}_{\langle I_c,F_c \rangle}(x_{k})). 
\end{align*}

This induces two types of the errors detailed above: first, the truncation
error due to representing $x_k$ as $\mathcal{F}{\langle I_c,F_c
\rangle}(x_{k})$; and second, the rounding error introduced by the
multiplication operation.  We represent these errors as non-deterministic
additive noise.

An additional error is due to the representation of the plant dynamics, namely 
%
\begin{align*}
x_{k+1} &=\mathcal{F}_{\langle I_p,F_p \rangle}(A_d) \mathcal{F}_{\langle I_p,F_p \rangle}(x_{k}) + \mathcal{F}_{\langle I_p,F_p \rangle}(B_d)\mathcal{F}_{\langle I_p,F_p \rangle}(u_{k}).
\end{align*}
We address this error by use of interval
arithmetic~\cite{moore1966interval} in the verification phase.

Previous studies~\cite{gangli1} show that the FWL affects the poles and
zeros positions, degrading the closed-loop dynamics, causing steady-state
errors (see Appendix~\ref{sec:appendix:LTIbackground} for details) and
eventually de-stabilizing the system~\cite{Bessa16}.  However, since in this
paper we require stability only as a precursor to safety, it is sufficient
to check that the (perturbed, noisy) model converges to a neighborhood of
the equilibrium within the safe set (see Appendix~\ref{sec:stab_FWL}).

In the following, we shall disregard these steady-state errors (caused by
FWL effects) when stability is ensured by synthesis, and then verify its
safety accounting for the finite-precision errors.


%-------------------------------
\section{Experimental Evaluation}
\label{exp:evaluation}
%-------------------------------


%-------------------------------
\subsection{Description of the benchmarks}
\label{exp:benchmarks}
%-------------------------------


%-------------------------------
\subsection{Objectives}
\label{exp:objectives}
%-------------------------------


%-------------------------------
\subsection{Results}
\label{exp:results}
%-------------------------------

%-------------------------------
\subsection{Threats to validity}
\label{exp:threats-to-validity}
%-------------------------------

%-------------------------------
\section{Conclusions}
\label{sec:conclusions}
%-------------------------------

\textcolor{red}{We need to write a new conclusions about this work.}

%\addtodo{[Move material from \ref{sec:rw} here]}


%\begin{ack}                               % Place acknowledgements
%Partially supported by the Roman Senate.  % here.
%\end{ack}

\bibliographystyle{plain}        % Include this if you use bibtex 
\bibliography{paper}           % and a bib file to produce the 
                                 % bibliography (preferred). The
                                 % correct style is generated by
                                 % Elsevier at the time of printing.

%\begin{thebibliography}{99}     % Otherwise use the  
                                 % thebibliography environment.
                                 % Insert the full references here.
                                 % See a recent issue of Automatica 
                                 % for the style.
%  \bibitem[Heritage, 1992]{Heritage:92}
%     (1992) {\it The American Heritage. 
%     Dictionary of the American Language.}
%     Houghton Mifflin Company.
%  \bibitem[Able, 1956]{Abl:56}
%     B.~C.~Able (1956). Nucleic acid content of macroscope. 
%     {\it Nature 2}, 7--9. 
%  \bibitem[Able {\em et al.}, 1954]{AbTaRu:54}   
%     B.~C. Able, R.~A. Tagg, and M.~Rush (1954).
%     Enzyme-catalyzed cellular transanimations.
%     In A.~F.~Round, editor, 
%     {\it Advances in Enzymology Vol. 2} (125--247). 
%     New York, Academic Press.
%  \bibitem[R.~Keohane, 1958]{Keo:58}
%     R.~Keohane (1958).
%     {\it Power and Interdependence: 
%     World Politics in Transition.}
%     Boston, Little, Brown \& Co.
%  \bibitem[Powers, 1985]{Pow:85}
%     T.~Powers (1985).
%     Is there a way out?
%     {\it Harpers, June 1985}, 35--47.

%\end{thebibliography}

\appendix


%-------------------------------
\section{Stability of Closed-loop Models}
\label{sec:appendix-stability}
%-------------------------------

%-------------------------------
\subsection{Stability of closed-loop models with fixed-point controller error}
\label{sec:stab_FWL}
%-------------------------------

The proof of Jury's criterion~\cite{fadali} relies on the fact that the relationship between states and
next states is defined by $x_{k+1} = (A_d - B_dK) x_k$, all computed
at infinite precision.  When we employ a FWL digital controller, the
operation becomes:
%
\begin{align*}
x_{k+1} &= A_d \cdot x_{k} -(\mathcal{F}_{\langle I_c,F_c \rangle}(K)\cdot\mathcal{F}_{\langle I_c,F_c \rangle}(x_{k})).  \\
x_{k+1} &= (A_d  - B_dK) \cdot x_k + B_dK\delta, 
\end{align*}
%
where $\delta$ is the maximum error that can be introduced by the FWL
controller in one step, i.e., by reading the states values once and
multiplying by $K$ once.  We derive the closed form expression for $x_n$ as
follows:
%
\begin{align*}
x_{1} &= (A_d  - B_dK)x_0 + B_dK\delta \\
x_{2} 
 &=(A_d  - B_dK)^2x_0 + (A_d  - B_dK)B_dK\delta + B_dK\delta \\
x_{n} &= (A_d  - B_dK)^nx_0 + (A_d  - B_dK)^{n-1}B_dK\delta + ... \\  \nonumber & + (A_d  - B_dK)^1B_dK \delta + B_dK\delta \\
  &= (A_d - B_dK)^nx_0 + \sum_{i=0}^{i=n-1}(A_d - B_dK)^iB_dk\delta. 
\end{align*}

The definition of asymptotic stability is that the system converges to a
reference signal, in this case we use no reference signal so an
asymptotically stable system will converge to the origin.  We know that the
original system with an infinite-precision controller is stable, because we
have synthesized it to meet Jury's criterion.  Hence, $(A_d - B_dK)^n x_0$ 
must converge to zero.

The power series of matrices
converges~\cite{horn1990matrix} iff the eigenvalues of the matrix are less
than~1 as follows:
%
%\begin{align*}
$\sum_{i=0}^{\infty}T^i  = (I - T)^{-1}$, 
%\end{align*}
%
where $I$ is the identity matrix and $T$ is a square matrix. Thus, our system will converge to the value 
%
\begin{align*}
0 + (I - A_d + B_dK)^{-1}B_dk\delta \,. 
\end{align*}
%
As a result, if the value $(I - A_d + B_dK)^{-1}B_dk\delta$ is within the
safe space, then the synthesized fixed-point controller results in a safe
closed-loop model.  The convergence to a finite value, however, will not
make it asymptomatically stable.

%+++++++++++++++++++++++++++++++++++++++++++++++++++++++++++++++++++++++++++++++
\section{Errors in LTI models} \label{sec:appendix:LTIbackground}
%+++++++++++++++++++++++++++++++++++++++++++++++++++++++++++++++++++++++++++++++

\subsection{Errors due to numerical representation} \label{appendix:numerical_errors}

We have  used $\mathcal{F}_{\langle I,F \rangle}(x)$ denote a real number
$x$ represented in a fixed point domain, with $I$ bits representing the
integer part and $F$ bits representing the decimal part.  The smallest
number that can be represented in this domain is $c_m=2^{-F}$.  The
following approximation errors will arise in mathematical operations and
representation:
%
\begin{enumerate}

\item {\bf Truncation:} Let $x$ be a real number, and $\mathcal{F}_{\langle
I,F \rangle}(x)$ be the same number represented in a fixed-point domain as
above.  Then $\mathcal{F}_{\langle I,F \rangle}(x) = x-\delta_T$ where the
error $ \delta_T=x\ \%_{c_m}\ \tilde x$, and $\%_{c_m}$ is the modulus
operation performed on the last bit. 
Thus, $\delta_T$ is the truncation error and it will propagate across
operations.
%
\item {\bf Rounding:} The following errors appear in basic operations.  Let
$c_1, c_2$ and $c_3$ be real numbers, and $\delta_{T1}$ and $\delta_{T2}$ be
the truncation errors caused by representing $c_1$ and $c_2$ in the
fixed-point domain as above.
%
\begin{enumerate}
%
\item Addition/Subtraction: these operations only propagate errors coming
from truncation of the operands, namely $\mathcal{F}_{\langle I,F
\rangle}(c_1) \pm \mathcal{F}_{\langle I,F \rangle}(c_2) = c_3 + \delta_3$
with $|\delta_3| \leq |\delta_{T1}| + |\delta_{T2}|$.
%
\item Multiplication: $\mathcal{F}_{\langle I,F \rangle}(c_1) \cdot
\mathcal{F}_{\langle I,F \rangle}(c_2) =  c_3 + \delta_3$ with $|\delta_3|
\leq |\delta_{T1}\cdot\mathcal{F}_{\langle I,F \rangle}(c_2)|\allowbreak +
|\delta_{T2}\cdot\mathcal{F}_{\langle I,F \rangle}(c_1)| + c_m$, where
$c_m=2^{-F}$ as above.
%
\item Division: the operations performed by our controllers in the FWL
domain do not include division.  However, we do use division in computations
at the precision of the plant.  Here the error depends on whether the
divisor is greater or smaller than the dividend:  $\mathcal{F}_{\langle I,F
\rangle}(c_1) / \mathcal{F}_{\langle I,F \rangle}(c_2) = c_3 + \delta_{T3}$
where $\delta_{T3}$ is $(\delta_{T2}\cdot c_1 - \delta_{T1}\cdot
c_2)/(\delta_{T2}^2 - \delta_{T2} c_2)$,
\end{enumerate}

\item {\bf Overflow:}
The maximum size of a real number $x$ that can be represented in a fixed
point domain as $\mathcal{F}_{\langle I,F \rangle}(x)$ is $\pm
(2^{I-1}+1-2^{-F})$.  Numbers outside this range cannot be represented by
the domain.  We check that overflow does not occur.

\end{enumerate}

%+++++++++++++++++++++++++++++++++++++++++++++++++++++++++++++++++++++++++++++++
\subsection{Modeling quantization as noise} \label{appendix:quantization-noise}
%+++++++++++++++++++++++++++++++++++++++++++++++++++++++++++++++++++++++++++++++

During any given ADC conversion, the continuous signal will be sampled in
the real domain and transformed by $\mathcal{F}\langle I_{c},F_{c} \rangle
(x)$ (assuming the ADC discretization is the same as the digital
implementation).  This sampling uses a threshold which is defined by the
less significant bit ($q_{c}=c_{m_c}=2^{-F_c}$) 
of the ADC and some non-linearities of the circuitry.  The overall conversion is
%
$$\mathcal{F}\langle I_{c},F_{c} \rangle(y(t)) = y_k :
y_k \in \left[y(t)-\frac{q_{c}}{2}\ \ \ \ y(t)+\frac{q_{c}}{2}\right] \,.$$
%
If we denote the error in the conversion by $\nu_k=y_k-y(t)$ where $t = nk$,
and $n$ is the sampling time and $k$ the number of steps, then we may define
some bounds for it $\nu_k \in [-\frac{q_{c}}{2}\ \ \frac{q_{c}}{2}]$.

We will assume, for the purposes of this analysis, that the domain of the
ADC is that of the digital controller (i.e, the quantizer includes any
digital gain added in the code).  The process of quantization in the DAC is
similar except that it is calculating $\mathcal{F}\langle I_{dac},F_{dac}
\rangle (\mathcal{F}\langle I_{c},F_{c} \rangle (x)) $.  If these domains
are the same ($I_{c}=I_{dac},\allowbreak F_{c}=F_{dac}$), or if the DAC
resolution in higher than the ADCs, then the DAC quantization error is equal
to zero.  From the above equations we can now define the ADC and DAC
quantization noises ${\nu_1}_k \in [-\frac{q_1}{2}\ \ \frac{q_1}{2}]$ and
${\nu_2}_k \in [-\frac{q_2}{2}\ \ \frac{q_2}{2}]$, where $q_1=q_{c}$ and 
$q_2=q_\mathit{dac}$.  This is illustrated in
Fig.~\ref{ssec:statefeedbackcontrol} where $Q_1$ is the quantizer of the ADC
and $Q_2$ the quantizer for the DAC.  These bounds hold irrespective of
whether the noise is correlated, hence we may use them to over-approximate
the noise effect on the state space progression over time.  The
resulting dynamics are
%
\begin{align*}
%\label{eq:pre_quantization}
{x}_{k+1} = {A}_d{x}_k+{B}_d({u}_k+{{\nu}_2}_k), \quad u_k = -K{x}_{k}+{{\nu}_1}_k, 
\end{align*}
%
which result in the following closed-loop dynamics:
%
\begin{align*}
%\label{eq:quantization}
{x}_{k+1} &= ({A}_d-{B}_d{K}_d) {x}_k+{B}_d{{\nu}_2}_k +{{\nu}_1}_k \,. 
\end{align*}



%\section{A summary of Latin grammar}    % Each appendix must have a short title.
%\section{Some Latin vocabulary}         % Sections and subsections are supported  
                                        % in the appendices.
\end{document}