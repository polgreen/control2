% autosam.tex
% Annotated sample file for the preparation of LaTeX files
% for the final versions of papers submitted to or accepted for 
% publication in AUTOMATICA.

% See also the Information for Authors.

% Make sure that the zip file that you send contains all the 
% files, including the files for the figures and the bib file.

% Output produced with the elsart style file does not imitate the
% AUTOMATICA style. The style file is generic for all Elsevier
% journals and the output is laid out for easy copy editing. The
% final document is produced from the source file in the
% AUTOMATICA style at Elsevier.

% You may use the style file autart.cls to obtain a two-column 
% document (see below) that more or less imitates the printed 
% Automatica style. This may helpful to improve the formatting 
% of the equations, tables and figures, and also serves to check 
% whether the paper satisfies the length requirements.

% Please note: Authors must not create their own macros.

% For further information regarding the preparation of LaTeX files 
% for Elsevier, please refer to the "Full Instructions to Authors" 
% from Elsevier's anonymous ftp server on ftp.elsevier.nl in the
% directory pub/styles, or from the internet (CTAN sites) on
% ftp.shsu.edu, ftp.dante.de and ftp.tex.ac.uk in the directory
% tex-archive/macros/latex/contrib/supported/elsevier.


%\documentclass{elsart}               % The use of LaTeX2e is preferred.

\documentclass[twocolumn]{autart}    % Enable this line and disable the 
                                     % preceding line to obtain a two-column 
                                     % document whose style resembles the
                                     % printed Automatica style.


\usepackage{graphicx}          % Include this line if your 
                               % document contains figures,
%\usepackage[dvips]{epsfig}    % or this line, depending on which
                               % you prefer.
\usepackage{color}

\begin{document}

\begin{frontmatter}
%\runtitle{Insert a suggested running title}  % Running title for regular 
                                              % papers but only if the title  
                                              % is over 5 words. Running title 
                                              % is not shown in output.

%\thanksref{footnoteinfo}

\title{Safe and Robust Formal  Synthesis of Digital Controllers for Continuous Plants with Transient Performance Specifications} % Title, preferably not more 
                                                % than 10 words.

%\thanks[footnoteinfo]{This paper was not presented at any IFAC 
%meeting. Corresponding author M.~T.~Cicero. Tel. +XXXIX-VI-mmmxxi. 
%Fax +XXXIX-VI-mmmxxv.}

\author[oxford]{Alessandro Abate}\ead{alessandro.abate@cs.ox.ac.uk},
\author[manaus]{Iury Bessa}\ead{iurybessa@ufam.edu.br},
\author[oxford]{Dario Cattaruzza}\ead{dario.cattaruzza@cs.ox.ac.uk},
\author[oxford,manaus]{Lucas Cordeiro}\ead{lucas.cordeiro@cs.ox.ac.uk},
\author[oxford]{Cristina David}\ead{cristina.david@cs.ox.ac.uk},
\author[oxford]{Pascal Kessel}\ead{pascal.kesseli@stx.ox.ac.uk},
\author[oxford]{Daniel Kroening}\ead{kroening@cs.ox.ac.uk},
\author[oxford]{Elizabeth Polgreen}\ead{elizabeth.polgreen@linacre.ox.ac.uk}

\address[oxford]{University of Oxford, UK}
\address[manaus]{Federal University of Amazonas, Brazil}

%\author[Paestum]{Marcus Tullius Cicero}\ead{cicero@senate.ir},    % Add the 
%\author[Rome]{Julius Caesar}\ead{julius@caesar.ir},               % e-mail address 
%\author[Baiae]{Publius Maro Vergilius}\ead{vergilius@culture.ir}  % (ead) as shown

%\address[Paestum]{Buckingham Palace, Paestum}  % Please supply                                              
%\address[Rome]{Senate House, Rome}             % full addresses
%\address[Baiae]{The White House, Baiae}        % here.

          
\begin{keyword}                           % Five to ten keywords,  
Digital Control; A/D converters; Control System Synthesis; Transient Analysis; Safety Analysis; Quantization Errors; Sampling.               % chosen from the IFAC 
\end{keyword}                             % keyword list or with the 
                                          % help of the Automatica 
                                          % keyword wizard

%\keywords{
%State-space dynamical models of physical systems; 
%digital controllers; 
%analogue-to-digital converters; 
%time sampling; 
%quantization; 
%fixed-point arithmetic; 
%CEGIS; 
%safety requirements. 
%}


\begin{abstract}                          % Abstract of not more than 200 words.
\textcolor{red}{We have to write a new abstract.}
\end{abstract}

\end{frontmatter}

\section{Introduction}
\textcolor{red}{We have to write a new introduction as well.} 

%\begin{figure}
%\begin{center}
%\includegraphics[height=4cm]{jcaesar.eps}    % The printed column  
%\caption{Gaius Julius Caesar, 100--44 B.C.}  % width is 8.4 cm.
%\label{fig1}                                 % Size the figures 
%\end{center}                                 % accordingly.
%\end{figure}

% OR

%\begin{figure}
%\begin{center}
%\epsfig{file=jcaesar,width=7cm}
%\caption{Gaius Julius Caesar, 100--44 B.C.}
%\label{fig1}
%\end{center}
%\end{figure}


%\subsection{A subsection}
%Marcus Tullius Cicero, 106--43 B.C. was a Roman statesman, orator, 
%and philosopher.  A major figure in the last years of the Republic, 
%he is best known for his orations against Catiline\footnote{
%This footnote should be very brief.}
%and for his mastery of Latin prose \cite{Heritage:92}. He was a 
%contemporary of Julius Caesar (Fig.~\ref{fig1}).

%-------------------------------
\section{Related Work}
\label{sec:relw}
%-------------------------------

\textbf{CEGIS -}
Program synthesis is the problem of computing correct-by-design programs
from high-level specifications. Algorithms for this problem have made
substantial progress in recent years, for instance~\cite{itzhaky2010simple} 
to inductively synthesize invariants for the generation of desired programs.

Program synthesizers are an ideal fit for the synthesis of digital controllers, since
the semantics of programs capture the effects of finite-precision arithmetic
precisely.  In~\cite{DBLP:conf/cdc/RavanbakhshS15}, the authors use CEGIS
for the synthesis of switching controllers for stabilizing continuous-time
plants with polynomial dynamics.  The work extends to affine systems, but is
limited by the capacity of the state-of-the-art SMT solvers for solving
linear arithmetic.  Since this approach uses switching models instead of
linear dynamics for the digital controller, it avoids problems related to
finite precision arithmetic, but potentially suffers from state-space
explosion.  Moreover, in \cite{DBLP:conf/emsoft/RavanbakhshS16} the same
authors use a CEGIS-based approach for synthesizing continuous-time
switching controllers that guarantee \emph{reach-while-stay} properties of
closed-loop systems, i.e., properties that specify a set of goal states and
safe states (constrained reachability).  This solution is based on
synthesizing control Lyapunov functions for switched systems that yield
switching controllers with a guaranteed minimum dwell time in each mode. 
However, both approaches are unsuitable for the kind of control we seek to
synthesize.

The work in~\cite{hscc-paper} synthesizes stabilizing
controllers for continuous plants given as transfer functions by exploiting
bit-accurate verification of software-implemented digital
controllers~\cite{Bessa16}.  While this work also uses CEGIS,
the approach is restricted to digital controllers for stable closed-loop
systems given as transfer function models: 
this results in  a static check on their coefficients.  
By contrast, in the current paper we consider a state-space representation of the physical system, 
which requires ensuring the specification over actual traces of the model, 
alongside the numerical soundness required by the effects of discretisation and finite-precision errors.   
A state-space model has known advantages over the transfer function
representation~\cite{Franklin15}: it naturally generalizes to multivariate systems
(i.e., with multiple inputs and outputs); 
and it allows synthesis of control systems with guarantees on the internal dynamics, e.g.,
to synthesize controllers that make the closed-loop system \emph{safe}.  Our
work focuses on the \emph{safety} of internal states, which is usually
overlooked in the literature.  Moreover, our work integrates an
abstraction/refinement (CEGAR) step inside the main CEGIS loop.

The tool Pessoa~\cite{mazo2010pessoa} synthesizes correct-by-design embedded
control software in a Matlab toolbox.  It is based on the abstraction of a
physical system to an equivalent finite-state machine and on the computation
of reachability properties thereon. 
Based on this safety specification, \mbox{Pessoa} can synthesize embedded controller
software for a range of properties.  The embedded controller software can be
more complicated than the state-feedback control we synthesize, and the
properties available cover more detail. 
However, relying on state-space discretization \mbox{Pessoa} is likely to incur in scalability limitations. 
Along this research line, \cite{Anta2010,liu16} studies the synthesis of digital controllers for continuous dynamics, 
and \cite{zamani2014} extends the approach to the recent setup of Network Control Systems. 

\textbf{Discretization Effects -}
The classical approach to control synthesis has often disregarded digitalization effects, 
whereas more recently modern techniques have focused on
different aspects of discretization, including delayed
response~\cite{Duggirala2015} and finite word length (FWL) semantics, 
with the goal either to verify (e.g.,~\cite{daes20161}) or to optimize
(e.g.,~\cite{oudjida2014design}) given implementations. 

There are two different problems that arise from FWL semantics.  The first
is the error in the dynamics caused by the inability to represent the exact
state of the physical system, while the second relates to rounding and saturation errors
during computation.  In~\cite{fialho1994stability}, a stability measure
based on the error of the digital dynamics ensures that the deviation
introduced by FWL does not make the digital system unstable.  A~more recent
approach~\cite{DBLP:journals/automatica/WuLCC09} uses $\mu$-calculus to
directly model the digital controller so that the selected parameters are
stable by design.  The analyses in~\cite{DBLP:conf/hybrid/RouxJG15,
DBLP:conf/hybrid/WangGRJF16} rely on an invariant computation on the
discrete system dynamics using Semi-Definite Programming (SDP).  While the
former uses bounded-input and bounded-output (BIBO) properties to determine
stability, the latter uses Lyapunov-based quadratic invariants.  In both
cases, the SDP solver uses floating-point arithmetic and soundness is
checked by bounding the error.  An alternative is~\cite{park2016scalable},
where the verification of given control code is performed against a known
model by extracting an LTI model of the code by symbolic execution:  
to account for rounding errors, an upper bound is introduced in the
verification phase.  The work in \cite{picasso2003stabilization}
introduces invariant sets as a mechanism
to bound the quantization error effect on stabilization as an invariant set
that always converges toward the controllable set.  Similarly,
\cite{liberzon2003hybrid} evaluates the quantization error dynamics
and bounds its trajectory to a known region over a finite time period. 
This technique works for both linear and non-linear systems.

%\addtodo{[Move material from \ref{sec:rw} here]}


%\begin{ack}                               % Place acknowledgements
%Partially supported by the Roman Senate.  % here.
%\end{ack}

\bibliographystyle{plain}        % Include this if you use bibtex 
\bibliography{paper}           % and a bib file to produce the 
                                 % bibliography (preferred). The
                                 % correct style is generated by
                                 % Elsevier at the time of printing.

%\begin{thebibliography}{99}     % Otherwise use the  
                                 % thebibliography environment.
                                 % Insert the full references here.
                                 % See a recent issue of Automatica 
                                 % for the style.
%  \bibitem[Heritage, 1992]{Heritage:92}
%     (1992) {\it The American Heritage. 
%     Dictionary of the American Language.}
%     Houghton Mifflin Company.
%  \bibitem[Able, 1956]{Abl:56}
%     B.~C.~Able (1956). Nucleic acid content of macroscope. 
%     {\it Nature 2}, 7--9. 
%  \bibitem[Able {\em et al.}, 1954]{AbTaRu:54}   
%     B.~C. Able, R.~A. Tagg, and M.~Rush (1954).
%     Enzyme-catalyzed cellular transanimations.
%     In A.~F.~Round, editor, 
%     {\it Advances in Enzymology Vol. 2} (125--247). 
%     New York, Academic Press.
%  \bibitem[R.~Keohane, 1958]{Keo:58}
%     R.~Keohane (1958).
%     {\it Power and Interdependence: 
%     World Politics in Transition.}
%     Boston, Little, Brown \& Co.
%  \bibitem[Powers, 1985]{Pow:85}
%     T.~Powers (1985).
%     Is there a way out?
%     {\it Harpers, June 1985}, 35--47.

%\end{thebibliography}

%\appendix
%\section{A summary of Latin grammar}    % Each appendix must have a short title.
%\section{Some Latin vocabulary}         % Sections and subsections are supported  
                                        % in the appendices.
\end{document}