% autosam.tex
% Annotated sample file for the preparation of LaTeX files
% for the final versions of papers submitted to or accepted for 
% publication in AUTOMATICA.

% See also the Information for Authors.

% Make sure that the zip file that you send contains all the 
% files, including the files for the figures and the bib file.

% Output produced with the elsart style file does not imitate the
% AUTOMATICA style. The style file is generic for all Elsevier
% journals and the output is laid out for easy copy editing. The
% final document is produced from the source file in the
% AUTOMATICA style at Elsevier.

% You may use the style file autart.cls to obtain a two-column 
% document (see below) that more or less imitates the printed 
% Automatica style. This may helpful to improve the formatting 
% of the equations, tables and figures, and also serves to check 
% whether the paper satisfies the length requirements.

% Please note: Authors must not create their own macros.

% For further information regarding the preparation of LaTeX files 
% for Elsevier, please refer to the "Full Instructions to Authors" 
% from Elsevier's anonymous ftp server on ftp.elsevier.nl in the
% directory pub/styles, or from the internet (CTAN sites) on
% ftp.shsu.edu, ftp.dante.de and ftp.tex.ac.uk in the directory
% tex-archive/macros/latex/contrib/supported/elsevier.


%\documentclass{elsart}               % The use of LaTeX2e is preferred.

\documentclass[twocolumn]{autart}    % Enable this line and disable the 
                                     % preceding line to obtain a two-column 
                                     % document whose style resembles the
                                     % printed Automatica style.


\usepackage{graphicx}          % Include this line if your 
                               % document contains figures,
%\usepackage[dvips]{epsfig}    % or this line, depending on which
                               % you prefer.
\usepackage{color}

\begin{document}

\begin{frontmatter}
%\runtitle{Insert a suggested running title}  % Running title for regular 
                                              % papers but only if the title  
                                              % is over 5 words. Running title 
                                              % is not shown in output.

%\thanksref{footnoteinfo}

\title{Safe and Robust Formal  Synthesis of Digital Controllers for Continuous Plants with Transient Performance Specifications} % Title, preferably not more 
                                                % than 10 words.

%\thanks[footnoteinfo]{This paper was not presented at any IFAC 
%meeting. Corresponding author M.~T.~Cicero. Tel. +XXXIX-VI-mmmxxi. 
%Fax +XXXIX-VI-mmmxxv.}

\author[abate]{Alessandro Abate}\ead{alessandro.abate@cs.ox.ac.uk},
\author[bessa]{Iury Bessa}\ead{iurybessa@ufam.edu.br},
\author[cattaruzza]{Dario Cattaruzza}\ead{dario.cattaruzza@cs.ox.ac.uk},
\author[cordeiro]{Lucas Cordeiro}\ead{lucas.cordeiro@cs.ox.ac.uk},
\author[david]{Cristina David}\ead{cristina.david@cs.ox.ac.uk},
\author[kesseli]{Pascal Kessel}\ead{pascal.kesseli@stx.ox.ac.uk},
\author[kroening]{Daniel Kroening}\ead{kroening@cs.ox.ac.uk},
\author[polgreen]{Elizabeth Polgreen}\ead{elizabeth.polgreen@linacre.ox.ac.uk}

\address[abate]{University of Oxford, UK}
\address[bessa]{Federal University of Amazonas, Brazil}
\address[cattaruzza]{University of Oxford, UK}
\address[cordeiro]{University of Oxford, UK}
\address[david]{University of Oxford, UK}
\address[kesseli]{University of Oxford, UK}
\address[kroening]{University of Oxford, UK}
\address[polgreen]{University of Oxford, UK}

%\author[Paestum]{Marcus Tullius Cicero}\ead{cicero@senate.ir},    % Add the 
%\author[Rome]{Julius Caesar}\ead{julius@caesar.ir},               % e-mail address 
%\author[Baiae]{Publius Maro Vergilius}\ead{vergilius@culture.ir}  % (ead) as shown

%\address[Paestum]{Buckingham Palace, Paestum}  % Please supply                                              
%\address[Rome]{Senate House, Rome}             % full addresses
%\address[Baiae]{The White House, Baiae}        % here.

          
\begin{keyword}                           % Five to ten keywords,  
Cicero; Catiline; orations.               % chosen from the IFAC 
\end{keyword}                             % keyword list or with the 
                                          % help of the Automatica 
                                          % keyword wizard


\begin{abstract}                          % Abstract of not more than 200 words.
\textcolor{red}{We have to write a new abstract.}
\end{abstract}

\end{frontmatter}

\section{Introduction}
\textcolor{red}{We have to write a new introduction as well.} 

%\begin{figure}
%\begin{center}
%\includegraphics[height=4cm]{jcaesar.eps}    % The printed column  
%\caption{Gaius Julius Caesar, 100--44 B.C.}  % width is 8.4 cm.
%\label{fig1}                                 % Size the figures 
%\end{center}                                 % accordingly.
%\end{figure}

% OR

%\begin{figure}
%\begin{center}
%\epsfig{file=jcaesar,width=7cm}
%\caption{Gaius Julius Caesar, 100--44 B.C.}
%\label{fig1}
%\end{center}
%\end{figure}


\subsection{A subsection}
Marcus Tullius Cicero, 106--43 B.C. was a Roman statesman, orator, 
and philosopher.  A major figure in the last years of the Republic, 
he is best known for his orations against Catiline\footnote{
This footnote should be very brief.}
and for his mastery of Latin prose \cite{Heritage:92}. He was a 
contemporary of Julius Caesar (Fig.~\ref{fig1}).

\section{The argument}
Some words might be appropriate describing equation~(\ref{e1}), if 
we had but time and space enough.
\begin{equation} \label{e1}
{{\partial F}\over {\partial t}} =
D{{\partial^2 F}\over {\partial x^2}}.
\end{equation}
See \cite{Abl:56}, \cite{AbTaRu:54}, \cite{Keo:58} and 
\cite{Pow:85}.
This equation goes far beyond the celebrated theorem ascribed to the great
Pythagoras by his followers.
\begin{thm}
The square of the length of the hypotenuse of a right triangle equals the sum of the squares 
of the lengths of the other two sides.
\end{thm}
\section{Epilogue}
A word or two to conclude, and this even includes some inline 
maths:  $R(x,t)\sim t^{-\beta}g(x/t^\alpha)\exp(-|x|/t^\alpha)$.

\begin{ack}                               % Place acknowledgements
Partially supported by the Roman Senate.  % here.
\end{ack}

\bibliographystyle{plain}        % Include this if you use bibtex 
\bibliography{paper}           % and a bib file to produce the 
                                 % bibliography (preferred). The
                                 % correct style is generated by
                                 % Elsevier at the time of printing.

%\begin{thebibliography}{99}     % Otherwise use the  
                                 % thebibliography environment.
                                 % Insert the full references here.
                                 % See a recent issue of Automatica 
                                 % for the style.
%  \bibitem[Heritage, 1992]{Heritage:92}
%     (1992) {\it The American Heritage. 
%     Dictionary of the American Language.}
%     Houghton Mifflin Company.
%  \bibitem[Able, 1956]{Abl:56}
%     B.~C.~Able (1956). Nucleic acid content of macroscope. 
%     {\it Nature 2}, 7--9. 
%  \bibitem[Able {\em et al.}, 1954]{AbTaRu:54}   
%     B.~C. Able, R.~A. Tagg, and M.~Rush (1954).
%     Enzyme-catalyzed cellular transanimations.
%     In A.~F.~Round, editor, 
%     {\it Advances in Enzymology Vol. 2} (125--247). 
%     New York, Academic Press.
%  \bibitem[R.~Keohane, 1958]{Keo:58}
%     R.~Keohane (1958).
%     {\it Power and Interdependence: 
%     World Politics in Transition.}
%     Boston, Little, Brown \& Co.
%  \bibitem[Powers, 1985]{Pow:85}
%     T.~Powers (1985).
%     Is there a way out?
%     {\it Harpers, June 1985}, 35--47.

%\end{thebibliography}

\appendix
\section{A summary of Latin grammar}    % Each appendix must have a short title.
\section{Some Latin vocabulary}         % Sections and subsections are supported  
                                        % in the appendices.
\end{document}