\documentclass{article}

\setlength\parindent{0pt}

\usepackage{amssymb,latexsym,amsfonts,amsmath, amsthm,color,mathrsfs}
\usepackage[dvipsnames]{xcolor}
\usepackage{amsthm}
\usepackage{mathtools}
\usepackage{dsfont}
\usepackage{mathtools}
\usepackage{todonotes}
%\usepackage{utf8}
\usepackage{fullpage}
\def\SS#1{{\textcolor{red}{ {\bf SS:} #1}}}
\renewcommand{\b}{\textcolor{blue}}
%\newcommand{\red}{\textcolor{red}}
%
\newcommand{\vold}[1]{\colorbox{red!20}{\bfseries #1}}
\newcommand{\vnew}[1]{\colorbox{blue!30}{\bfseries #1}}

\renewcommand{\t}{\new{(t)}}
\newcommand{\tp}{\new{(t+1)}}
\newcommand{\fei}[1]{{\color{purple}#1 \color{black}}}

\begin{document}
\begin{center}
{\Large Responses to the Comments}\\\vspace{0.3cm}
\today
\end{center}
\vspace{1cm}
Dear Associate Editor and Reviewers,\vspace{0.3cm}
\\
We would like to deeply thank you for reviewing our paper and providing us with the constructive comments and suggestions. We have carefully addressed
all the comments and incorporated them in the revised manuscript.

We present below a detailed discussion of all comments, questions, and concerns raised by the reviewers. The major modifications are highlighted in
blue in the revised manuscript. For clarity,  we also use \vold{v1} and \vnew{v2} in the response letter for referring respectively to the previous
and current versions of the manuscript.
%I do not like color coding that much. And definitely not with red and blue.
\\
\\
With kind regards,\vspace{0.1cm}
\\

%The Authors
%
\newpage

\section{General response}




\section{Response to Reviewer (1)}
\bigskip

{\bf Comment: }{\itshape This paper presents an approach for synthesizing 
digital/software controllers for continuous plants (linear time invariant / LTI) 
such that they satisfy safety/stability requirements, as would show up in closed-loop 
control of cyber-physical systems (CPS). The approach extends earlier work of the authors, 
the primary of which in my view is toward floating-point implementation as opposed to only fixed-point. 
I did not entirely buy that using CEGIS is a contribution, as from looking over the earlier related publications 
(e.g., ASE'17, CAV'17, HSCC'17, SYNT'18 abstract), all of them use CEGIS.}

\vspace{1em}

{\bf Response: }

Do we actually say that CEGIS is a contribution? We should change that

\todo[inline]{check that we don't say CEGIS is a contribution}

\vspace{2em}
{\bf Comment: } {\itshape he results are evaluated on about a dozen and a half examples from the controls literature all with relatively small
dimensionality (2 to 5 real variables in the continuous state space), and while the authors claim the software artifacts and benchmarks are available
for evaluation, when this reviewer browsed to the link (http://www.cprover.org/DSSynth/), a page with "TBD" was shown. I did a little digging and
eventually found what I presume is the same / similar artifacts here: https://github.com/ssvlab/dsverifier and/or http://dsverifier.org/}

\vspace{1em}
{\bf Response: }

Fixed this now

\todo[inline]{Elizabeth: fix this}
\vspace{2em}
{\bf Comment: } {\itshape On the control theory side, the approach uses standard results to synthesize stabilizing controllers for LTI systems,
specifically linear state feedback controllers (u=Kx with a possible reference signal to track), under the assumption the plant is controllable. It
seems the approach requires full-state observability as well, which is a fairly standard assumption as one can typically implement various filters /
estimators (Kalman, Leuenberger, etc.) to ensure state observability assuming detectability of the plant, but the authors should clarify this point to
make the assumptions more precise.}
\vspace{1em}
{\bf Response: }

We have clarified this on page [X] as follows: 

\todo[inline]{fix this}
\vspace{2em}
{\bf Comment: } {\itshape Theorem 2 states completeness of the approach, specifically that there exists a finite k for which BMC can be done up-to to
provide a guarantee of future safety for all larger k' > k. Assuming the benchmarks all work up to some completeness threshold, I would like to see
the completeness threshold for each benchmark in table 1. The reason being, I find this result a bit surprising in the given context (albeit not
unbelievable given the focus on LTI) and would like to see further evidence illustrating the point beyond the fairly informal proof, and it would
greatly enhance the existing discussion of timeouts, etc. related to completeness thresholds being too high for some cases.}

\vspace{1em}

{\bf Response: }

We have added the completeness threshold for each benchmark in the table. Note that the completeness threshold is dependent on the controller, 
and a controller with a smaller completeness threshold may exist. 

The proof has been elaborated on as follows:

\todo[inline]{Elizabeth: add completeness thresholds to table}
\todo[inline]{Alessandro: Elaborate and make more formal Theorem 2}

\vspace{2em}

{\bf Comment: }
{\itshape I would like to see some additional discussion and/or comparison to hybrid systems reachability. While the focus of this work is on lower-level
software synthesis, if the approach is to be believed in the context of closed-loop systems satisfying specifications, then I would like some
discussion/comparison to that vast existing literature. Specifically, I presume the reason the benchmarks are not scalable beyond ~n=4 dimensionality
(as the n=5 benchmark appears to timeout), which is trivially small at this stage for linear systems (e.g., given a recent HSCC'19 paper on verifying
$R^{billion}$ linear systems, or at least n=100 to n=1000 with standard approaches like SpaceEx), is due to overapproximation error growth with the
interval representation of plant behaviors, whereas most scalable approaches utilize better data structures for representing geometric sets
(zonotopes, support functions, star sets, Taylor models, etc.). This likewise limits the possible applicability of the
approach for nonlinear plants, although there are a host of technical issues in that potential extension. }

\vspace{1em}
{\bf Response: }


\vspace{2em}
{\bf Comment: }
{\itshape Finally, while checking the $2^n$ vertices
are sufficient (theorem 1), I doubt this is necessary, as e.g. the star set approaches avoid this blow-up that would also limit the dimensionality
the approach can work toward.}
\vspace{1em}


{\bf Response: }
\todo[inline]{Alessandro: response to this (Theorem 1)}

\vspace{2em}

{\bf Comment: }
{\itshape While I may have missed it, I didn't see how the controller gain K is initially selected. I presume under the given assumptions a standard approach
such as pole placement, but this should be clarified}
\vspace{1em}
{\bf Response: }

The controller gain is initially nondeterministically selected by the bounded model checking synthesis step, i.e., the SAT solver returns the first assignment to the controller
values it finds
that satisfies the boolean formula which encodes the specification, by using Conflict Driven Clause Learning.
We have added the following sentence to clarify this:
\todo[inline]{add this}

\vspace{2em}
\todo[inline]{fix reviewer 1 typos}

\section{Response to Reviewer (3)}

{\bf Comment: }{\itshape
The verification condition, given in Theorem 1, is wrong. Here is an example system, with dynamics, initial set, and unsafe set.

\begin{verbatim}
x+ = -0.5*y
y+ = 0.5*x
\end{verbatim}


Rotation by 90 degrees in counterclock wise and shrinking it to the origin by 0.5.
\begin{verbatim}
initial set is 0 <= x <= 1 and 0 <= y <= 1.
unsafe set -3/20 <= x <= -2/20 and -3/20 <= y <= -2/20.
\end{verbatim}

It is easy to see that after two transformations, the initial set would be
\begin{verbatim}
-0.25 <= x <= 0 and -0.25 <= y <= 0.
\end{verbatim}
The unsafe set lies completely within the reachable set, but none of the edges in the hyperrectangle is unsafe. The next time the initial set reaches
this quadrant, the reachable set is
\begin{verbatim}
-1/(2^6) <= x <= 0
-1/(2^6) <= y <= 0.
\end{verbatim}

I believe that for this case, the K returned from completeness is 4 or 5.

The same theorem is also present in [1]. The proof, upon careful inspection, is not a proof of the statement given in the theorem. It only proves
that the reachable set is the convex set obtained by the trajectories of the corner states. However, it does not prove that if the trajectories are
safe, their convex combination is also safe. In fact, as given in the above counterexample, the vertices of polytope are safe does not mean that the
convex combination is safe.

I think this bug needs to be fixed in the BMC part of the verifier. Instead of checking the individual trajectories, you should in fact compute the
convex combination of trajectories (reachable set) and then perform the verification.}
\vspace{1em}
{\bf Response: }
\todo[inline]{Alessandro: Response to this (Theorem 1)}
\vspace{2em}

\section{Response to Reviewer (4)}

{\bf Comment:  } {\itshape The safety property in Eqn 3 focuses on interval constraints. One can generate auxiliary variables such that range constraints over them can encode
more complex safety properties. It would be nice to bring this up in the discussion after equation 3 that $x_i$ need not be just $i$-th component of the
state.  Can these also incorporate temporal properties (say STL properties with bounded time-horizon) without having to unroll the transition and
create auxiliary variables that track history explicitly?}
\vspace{1em}
{\bf Response: }

\vspace{2em}
{\bf Comment: } {\itshape The paper would benefit with a restructuring of the discussion. Figure 2 and 3 will be more useful
earlier in the paper (Section 1 or 3).}

\vspace{1em}
{\bf Response: }
We have moved these figures to [] 
\todo[inline]{move figure}
\vspace{2em}

{\bf Comment: } {\itshape
Some other relevant applications of CEGIS to CPS/dynamical systems include: Synthesizing switching logic for
safety and dwell-time requirements in ICCPS, 2010; Synthesis of optimal switching logic for hybrid systems in EMSOFT, 2011, Synthesis of Optimal
Fixed-Point Implementation of Numerical Software Routines in NSV'13, Bridging boolean and quantitative synthesis using smoothed proof search,
POPL'14, Synthesis of fixed-point programs in EMSOFT'13}

\vspace{1em}
{\bf Response: }
We have added a discussion about these applications and results to the related work as follows:
\todo[inline]{add discussion}

\vspace{2em}
{\bf Comment: } {\itshape Could you add one example to the end of Section 4 to show $R_1$, $R_2$, $R_3$, $R_4$ for a simple LTI system? Section 4 is one the most interesting parts
of the paper and its current presentation is a bit sketchy. What are $m_{ij}$?}
\vspace{1em}
{\bf Response: }



\bibliographystyle{alpha}
\bibliography{}




\end{document}
