
%+++++++++++++++++++++++++++++++++++++++++++++++++++++++++++++++++++++++++++++++
\subsection{Observable canonical form} \label{sec:observable}
%+++++++++++++++++++++++++++++++++++++++++++++++++++++++++++++++++++++++++++++++
The Controllable Canonical Form is ideal when we have access to the state space. However, in most cyberphisical
systems, only the outputs of the plant can be observed, hence we require an Observable Canonical Form.
For SISO systems, this is:
\begin{align}
\label{of_SISO}
\mat{A}_{of}=&\left[
\begin{array}{ccccc}
0&0&0&\cdots&-a_n\\
1&0&0&\cdots&-a_{n-1}\\
0&1&0&\cdots&-a_{n-2}\\
\vdots&\ddots&\ddots&\ddots&\vdots\\
0&0&\cdots&1&-a_1
\end{array}\right],
\mat{B}_{of}=\left[
\begin{array}{c}
\eta_n\\ \eta_{n-1}\\ \vdots\\ \eta_1
\end{array}\right]\\
\mat{C}_{of}=&[\begin{array}{ccccc}0&0&\cdots&0&1\end{array}] \nonumber
\end{align}

What is important to notice her is that $\mat{A}_{of}=\mat{A}_{cf}^T, \mat{C}_{of}=\mat{B}_{cf}^T$
and $\mat{B}_{of}=\mat{C}_{cf}^T$, which also applies to the MIMO case.

Let 
\begin{equation}
\label{eq:wnof}
\mat{W}_o=[\begin{array}{ccccc}\mat{C}&\mat{A}\mat{C}&\mat{A}^2\mat{C}&\hdots&\mat{C}^{n-1}\mat{B}\end{array}]^T
\end{equation}
be the observability matrix of the system, and
\begin{equation}
\label{eq:wof}
\mat{W}_{of}=\mat{R}_{cf}^{-1}
\end{equation}
the observability matrix of the canonical form (see eg~\cite{astrom1997computer}). 
Then the dynamical system 
\begin{equation}
\label{eq:to_of}
\hat{\vec{x}}_{k+1}=(\mat{A}_o-\mat{L}\mat{C}_o)\hat{\vec{x}}_k+\mat{B}_o\vec{u}+\mat{L}\vec{y} : \mat{L}=\mat{W}_o^{-1}\mat{W}_{of} \left[ \begin{array}{c}p_1-a_1\\p_2-a_2\\p_n-a_n\end{array}\right]
\end{equation}
\begin{displaymath}
\mat{A}_o=\mat{A},\ \mat{B}_o=\mat{B},\ \mat{C}_o=\mat{C}
\end{displaymath}
is an observer of the system $\vec{x}_{k+1}=\mat{A}\vec{x}+\mat{B}\vec{u} : \vec{y}=\mat{C}\vec{x}$. $p_i$ are user selected
values which will affect the error of the observer estimator. The differentiation between $\mat{*}_o$ and $\mat{*}$ will be exploited later in this work. For the time being we will assume they are the same.

Figure \ref{fig:observersystem} shows an LTI system with a feedback control using an observer. Presuming a sampled model of the plant as described in \eqref{eq:discretize}, the closed loop dynamics of this system are given by:
\begin{equation}
\left [\begin{array}{c}\vec{x}\\ \hat{\vec{x}}_e \end{array}\right]_{k+1}
=\left [\begin{array}{cc}\mat{A}_d-\mat{B}_d\mat{K}_d&\mat{B}_d\mat{K}_d\\ \mat{0}&\mat{A}_d-\mat{L}_d\mat{C}_d\end{array}\right]
\left [\begin{array}{c}\vec{x}\\ \hat{\vec{x}}_e \end{array}\right]_k
+\left [\begin{array}{c}\mat{B}_dk_r\\ \mat{0} \end{array}\right] \vec{r}_k
+\left [\begin{array}{cc}\mat{B}_dk_r&\mat{0}\\ \mat{0}&-\mat{L}_d\end{array}\right]\left [\begin{array}{c}\vec{\nu}_2\\ \vec{\nu}_1\end{array}\right]
\label{eq:observer_LTI}
\end{equation}
where $\hat{\vec{x}}_e=\vec{x}-\hat{\vec{x}}$ is the observer error.
Replacing the matrices in \eqref{eq:observer_LTI} with appropriate symbols, we find that it still has a similar structure to \eqref{eq:discretization} which will be helpful in our analysis.


\begin{figure*}[htb]
\centering

\tikzset{add/.style n args={4}{
    minimum width=6mm,
    path picture={
        \draw[circle] 
            (path picture bounding box.south east) -- (path picture bounding box.north west)
            (path picture bounding box.south west) -- (path picture bounding box.north east);
        \node[draw=none] at ($(path picture bounding box.south)+(0,0.13)$)     {\small #1};
        \node[draw=none] at ($(path picture bounding box.west)+(0.13,0)$)      {\small #2};
        \node[draw=none] at ($(path picture bounding box.north)+(0,-0.13)$)    {\small #3};
        \node[draw=none] at ($(path picture bounding box.east)+(-0.13,0)$)     {\small #4};
        }
    }
 }

\resizebox{1.0\textwidth}{!}{
 \begin{tikzpicture}[scale=0.6,-,>=stealth',shorten >=.2pt,auto,
     semithick, initial text=, ampersand replacement=\&,]

  \matrix[nodes={draw, fill=none, shape=rectangle, minimum height=.2cm, minimum width=.2cm, align=center}, row sep=.6cm, column sep=.6cm] {
    \node[draw=none] (r) {$r_k$};
    \node[rectangle,draw,
	minimum width=1cm,
	minimum height=1cm,] (gain)at ([xshift=1.2cm]r) {\sc $k_r$};
    
   \& \node[circle,add={-}{+}{}{}] (circle) {};
   \node[draw=none] (ez) at ([xshift=1cm,yshift=.15cm]circle)  {$u_k$};
   \node[rectangle,draw,minimum width=1cm,minimum height=1cm] (Kd) at ([xshift=0,yshift=-5.5cm]circle)  {\sc $\mat{K}_d$};
   \coordinate (kdsouth) at ([yshift=-3cm]Kd);
      
   
   \& complexnode/.pic={ 
      \node[rectangle,dashed,draw,minimum width=3cm,minimum height=1.6cm,label=\textbf{DAC}] (dac) {};
     \node[circle,add={+}{+}{}{},fill=gray!20] (q2) at ([xshift=-.65cm]dac.center) {};
     \node[draw=none] (q2t)  at ([yshift=.55cm]q2) {{\sc Q2}};
     \node[draw=none] (v2)  at ([yshift=-1.5cm]q2) {$\nu_2$};
     \node[fill=gray!20] (zoh) at ([xshift=.65cm]dac.center) {\sc ZOH};
   }   

   \& complexnode/.pic={ 
      \node[rectangle,dashed,draw,minimum width=8cm,minimum height=3.5cm,label=\textbf{Plant}] (plant)  at ([yshift=-.5cm]dac.center) {};
      \node[rectangle,draw,minimum width=1cm,minimum height=1cm] (B) at ([xshift=-2.5cm,yshift=.5cm]plant.center)  {\sc $\mat{B}$};
      \node[draw=none] (u) at ([xshift=-1cm,yshift=.15cm]B)  {$u(t)$};
      \node[circle,add={+}{+}{}{}] (p1) at ([xshift=-1.3cm,yshift=.5cm]plant.center) {};
      \node[draw=none] (xdot) at ([xshift=.85cm,yshift=.15cm]p1)  {$\dot{\vec{x}}(t)$};   
      \node[rectangle,draw,minimum width=1cm] (int) at ([xshift=.5cm,yshift=.5cm]plant.center) {\sc $\int$};
      \coordinate (xsouth) at ([xshift=1cm]int);
      \node[draw=none] (x) at ([xshift=1cm,yshift=.15cm]int)  {$\vec{x}(t)$};   
      \node[rectangle,draw,minimum width=1cm,minimum height=1cm] (C) at ([xshift=2.7cm,yshift=.5cm]plant.center)  {\sc $\mat{C}$};
      \node[draw=none] (y) at ([xshift=1cm,yshift=.15cm]C)  {$\vec{y}(t)$};   
      \node[rectangle,draw,minimum width=1cm,minimum height=1cm] (A) at ([xshift=.5cm,yshift=-1cm]plant.center)  {\sc $\mat{A}$};
      \coordinate (aeast) at ([xshift=1cm]A);
      \coordinate (awest) at ([xshift=-1.8cm]A);
    }   
    
   complexnode/.pic={ 
      \node[rectangle,dashed,draw,minimum width=8cm,minimum height=5cm,label=\textbf{Observer}] (observer)  at ([yshift=-5cm]plant.center) {};
      \node[rectangle,draw,minimum width=1cm,minimum height=1cm] (Bd) at ([xshift=-2.5cm]observer.center)  {\sc $\mat{B}_d$};
      \node[draw=none] (ud) at ([xshift=-1cm,yshift=.15cm]Bd)  {$u_k$};
      \node[circle,add={+}{+}{+}{}] (o1) at ([xshift=-1.3cm]observer.center) {};
      \node[draw=none] (xd) at ([xshift=.85cm,yshift=.15cm]o1)  {$\hat{\vec{x}}_{k+1}$};   
      \node[rectangle,draw,minimum width=1cm] (delay) at ([xshift=.5cm]observer.center) {\sc $z^{-1}$};
      \coordinate (xdsouth) at ([xshift=1.3cm]delay);
      \coordinate (fbsouth) at ([xshift=1.3cm,yshift=-3cm]delay);
      \node[draw=none] (xdp) at ([xshift=1cm,yshift=.15cm]delay)  {$\hat{\vec{x}}_k$};   
      \node[rectangle,draw,minimum width=1cm,minimum height=1cm] (Cd) at ([xshift=2.7cm]observer.center)  {\sc $\mat{C}_d$};
      \node[draw=none] (y) at ([xshift=.4cm,yshift=.8cm]Cd)  {$\hat{\vec{y}}_k$};   
      \node[rectangle,draw,minimum width=1cm,minimum height=1cm] (Ad) at ([xshift=.5cm,yshift=-1.5cm]observer.center)  {\sc $\mat{A}_d$};
      \coordinate (adeast) at ([xshift=1.3cm]Ad);
      \coordinate (adwest) at ([xshift=-1.8cm]Ad);
      \node[circle,add={-}{}{}{+}] (o2) at ([xshift=2.7cm,yshift=1.5cm]observer.center) {};
      \node[rectangle,draw,minimum width=1cm,minimum height=1cm] (Ld) at ([xshift=.5cm,yshift=1.5cm]observer.center)  {\sc $\mat{L}_d$};
      \coordinate (ldwest) at ([xshift=-1.8cm]Ld);
      \coordinate (bdwest) at ([xshift=-5.5cm]Bd);
      \coordinate (uksouth) at ([xshift=-5.5cm,yshift=5.5cm]Bd);
      \coordinate (o2east) at ([xshift=6.3cm]o2);
    }   
    
   \& complexnode/.pic={ 
     \node[rectangle,dashed,draw,minimum width=3.5cm,minimum height=1.6cm,label=\textbf{ADC},] (adc) {};
     \draw[] ([xshift=-1cm]adc.center) -- ++(0.5,0.2cm);
     \coordinate (switch1) at ([xshift=-1cm]adc.center);
     \coordinate (switch2) at ([xshift=-0.4cm]adc.center);
     \node[circle,add={+}{+}{}{},fill=gray!20] (q1) at ([xshift=.6cm]adc.center) {};
     \node[draw=none] (q2t)  at ([yshift=.55cm]q1) {\sc Q1};
     \node[draw=none] (v1)  at ([yshift=-1.5cm]q1) {$\nu_1$};
     \node[draw=none] (y) at ([xshift=.85cm,yshift=.15cm]q1)  {$\vec{y}_k$};       
     \coordinate (ykeast) at ([xshift=1.9cm]q1);
   } 
   \& \coordinate (aux1);
   \& \\
  };

  \path[->] (v1) edge (q1.south);
  \path[->] (v2) edge (q2.south);
  \path[->] (r) edge (gain.west);
  \path[->] (gain.east) edge (circle.west);
  \path[->] (circle.east) edge (q2.west);
  \path       (q2.east) edge (zoh.west);
  \path[->] (zoh.east) edge (B.west);
  \path
   (B.east) edge (p1.west)
   (p1.east) edge (int.west)
   (xsouth) edge (aeast)
   (aeast) edge (A.east)
   (A.west) edge (awest)
   (awest) edge (p1.south)
   (int.east) edge (C.west)
   (C.east) edge (switch1.west)
   (switch2) edge (q1.west);
  \path
   (q1.east) edge (ykeast)
   (ykeast) edge (o2east);
  \path       (uksouth) edge (bdwest);
  \path[->] (bdwest) edge (Bd.west);
  \path
   (Bd.east) edge (o1.west)
   (o1.east) edge (delay.west)
   (xdsouth) edge (adeast)
   (adeast) edge (Ad.east)
   (Ad.west) edge (adwest)
   (adwest) edge (o1.south)
   (o2.west) edge (Ld.east)
   (Ld.west) edge (ldwest)
   (ldwest) edge (o1.north)
   (delay.east) edge (Cd.west)
   (Cd.north) edge (o2.south)
   (o2east) edge (o2.east)
   (xdsouth) edge (fbsouth)
   (fbsouth) edge (kdsouth);
  \path[->]  (kdsouth) edge (Kd.south);
  \path (Kd.north) edge (circle.south);
 \end{tikzpicture}
}
 \caption{Closed-loop digital control system with observer\label{fig:observersystem}}
\end{figure*}